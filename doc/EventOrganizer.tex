\documentclass[oneside, a4, titlepage]{scrartcl}

\usepackage{graphicx}
\usepackage[american]{babel}
\usepackage{umlaute}
\usepackage{picture}
\usepackage{tabularx}
\usepackage{trfsigns}
\usepackage{prettyref}
\usepackage{titleref}
\usepackage{nameref}
\usepackage{subfig}
\usepackage{bm}
\usepackage{adjustbox}
\usepackage{varwidth}
\usepackage{amsmath,amssymb}
\usepackage{amsthm}
\usepackage[colorlinks, pdfpagelabels, pdfstartview = FitH, bookmarksopen = true, bookmarksnumbered = true,
linkcolor = black, plainpages = false, hypertexnames = false, citecolor = black, pagebackref = true, pdftex, pdfauthor={Berkay K�ksal, Alexander K�chler, Saad Lamdouar}, pdftitle={Ready2Meet: Event Organizer}] {hyperref}

\titlehead{\centering\includegraphics[width=6cm]{graphics/EurecomLogo.png}}
\title{MobServ Challenge Project 2017}
\subtitle{Ready2Meet: Event Organizer for Android}
\author{Berkay K\"oksal \and Alexander K\"uchler \and Saad Lamdouar}
\date{\today}

\renewcommand{\arraystretch}{1.25}
\newcolumntype{C}{>{\centering\arraybackslash}X}

\usepackage{enumitem}
\setlist{itemsep=0.1em}

\begin{document}
\maketitle

%\setcounter{tocdepth}{2}
%\tableofcontents
%\newpage

\section{Application Description}
The event organizer (name Ready2Meet) is an Android app to facilitate creating events and inviting people. The core of the application, namely organizing an event, can be used in two different modes:
\begin{enumerate}
\item Planned event
\item Spontaneous event
\end{enumerate}

For both modes, the user creates an event (including a start and end time, location, type of event and eventually a name). In the planned event mode, the user can invite specific people (friends) while for the spontaneous event, it is also possible to send an invitation to people which are close to the user's current location. Like this, he is able to spontaneously gather people based on their location.

Apart from this, all people joining an event should be able to share pictures of the event in the app and influence the event's outcome e.g. by voting for music on a party. Also, a chatroom should be available to communicate with the people before or after the event.

For all kinds of events, the organizing person can add a list of required material (e.g. drinks or food) so that the guests can register to bring some of the material and thus easing the organization.

Finally, every user invited to an event is provided additional information like the weather forecast for outdoor events and whether he is available on the date when the event is scheduled.

\section{Features}
The features of our event organization app include
\begin{itemize}
\item Create an event (e.g. party, sports, hiking, lunch, ...)
	\begin{itemize}
	\item Invite friends
	\item Invite people in a close area. I.e., the user can specify a radius to invite people and the people are notified if they are interested in this kind of events
	\end{itemize}
\item Chatroom for event
\item Enable music organization/voting during the event
\item Share pictures of the event
\item Who brings what?
\item Get some additional information for the event 
	\begin{itemize}
	\item Weather forecast for outdoor events
	\item Check if timeslot is free in your calendar
	\end{itemize}
\item Share the event in other networks e.g. on facebook
\end{itemize}

We therefore plan to use Google API (maps), firebase database to provide a real time database, and OpenWeatherMap API to retrieve the weather forecast.

\section{Developer Team}
The team consists of three people:
\begin{itemize}
\item Saad Lamdouar: UI developer
\item Berkay K\"oksal: Core developer
\item Alexander K\"uchler: Core developer
\end{itemize}

\section{Business Model}
Different business models are possible to push the app on the market. In this section, we discuss potential models and pricing strategies and finally chose the most promising model.

A completely free (e.g. open source) app is the first possibility. While this potentially targets the highest number of users as the app is free to use and accessible also in alternative stores like F-Droid, the application would not lead to any profit.

A second option is to use advertisements in the app in order to gain from the application. While this leads to a higher income, it contradicts the open-source strategy and thus would probably lead to a smaller target group. However, as the vast majority of Android users install apps from the Google Play Store, the decrease of users is acceptable.

Together with the app using advertisement, it is possible to offer an for a small price and without ads. An app which has only a priced version does not appear to be promising to us as most users will not be willing to pay for the service.

Also, In-App sales would be possible e.g. to enhance the possibilities of a user when creating events.
In especial, this makes sense for business customers that could use the application to advertise their business. A possible scenario is creating events in the application which reflects the companies product portfolio (e.g. guided tours, sports, parties in a bar, \dots). Like this, the company can acquire new customers which are nearby e.g. during their holidays and thus our app presents a new marketing strategy. In the scenario of commercial users, it would be possible to earn a reward for every user which joins the respective event due to our platform. The reward could be negotiated as a percentage of the tickets for the event or as a fixed amount. However, both cases would require to include a ticket-selling system for the billing. This would be a possible extension of the app.

In our case, the most promising strategy is to offer a free app with advertisements as well as a paid version without ads. In-App sales which target commercial customers of the app can make the app more powerful. E.g. an event could be permanent or the radius to invite people could be extended. Also, the creators of an event could have additional possibilities.

\end{document}
